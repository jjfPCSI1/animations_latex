\documentclass{standalone}

\usepackage{amsmath,amssymb,amsfonts}
\usepackage{graphics}

%%%%%%%%%%%%%%%%%%%%%%%%%%%%%%%%%%%%%%%%%%%%%
%             Les macros � JJ
%
% Quelques macros utilis�es dans mes documents
%
%%%%%%%%%%%%%%%%%%%%%%%%%%%%%%%%%%%%%%%%%%%%%

\newcommand{\bk}{\bigskip}
\newcommand{\ph}{\varphi}

\newcommand{\abs}[1]{\left|#1\right|}
\newcommand{\co}[1]{\underline{#1}}
\newcommand{\e}[1]{_{\text{#1}}}
\newcommand{\vect}[1]{\overrightarrow{#1}}

\newcommand{\f}[2]{\dfrac{#1}{#2}}


\usepackage[tikz]{bclogo}
\usepackage{tikz} % la base
\usetikzlibrary{calc,through,backgrounds,patterns}
\usepackage[babel=true,kerning=true]{microtype}

\usepackage{animate} % Pour faire les animations proprement dites

\begin{document}

% #1 -> longueur du ressort
% #2 -> symb�le associ� ($\ell_0$ par exemple)
\newcommand{\ressort}[2]{
\draw[ressort,decorate,decoration={coil,aspect=0.4,segment length = #1 mm,amplitude=3mm}] 
(0,-0.3)--(0,-#1) ;
\draw[thick,gray] (0,0) -- (0,-0.3); 
\fill [pattern=north east lines] (-0.5,0) rectangle (0.5,0.3); %bloc qui tient le ressort
\draw[thick] (-0.5,0) -- (0.5,0); %bloc qui tient le ressort
%fin ressort
\draw[line width=0.5pt,<->](-0.8,0) -- (-0.8,-#1) node 
[midway,left] {#2} ;
}%
%
% #1 -> longueur du ressort
% #2 -> symb�le associ� ($\ell\e{�q}$ par exemple)
% #3 -> Nom de la tension ($\vect{T\e{�q}}$ par exemple)
\newcommand{\masseressort}[3]{
\ressort{#1}{#2}
\draw [rounded corners=4pt,color=white,ball color=gray,smooth] (0,{-#1-\RAYON}) circle 
(\RAYON) ;
\draw [very thick,-latex] (0,{-#1-\RAYON})--++(0,-\POIDS) node [midway,right=0cm] 
{$\vect{P}$};
\draw [very thick,-latex] (0,{-#1-0.05})--(0,{-#1 + \K*(#1-\ELLZERO)-0.05}) node [midway,right=0.25cm] 
{#3};
}
%
%
\begin{animateinline}[autoplay,loop,controls]{10}  
    \multiframe{36}{rangle=0+10}{
\begin{tikzpicture} [scale=0.75, decoration={coil,aspect=0.4,segment 
length=3mm,amplitude=3mm}]
%decoration : g�re l'aspect du ressort par l'instruction decorate
\tikzset{ressort/.style={thick,gray,smooth}} %d�finition d'un style de ressort

\draw [dashed] (2,-4.5) --++ (4,0);
\draw node [left] at (2,-4.5) {O}; 
\draw [->] (2,-6.5) --++ (0,3) node [left] {$z$};
\draw [dashed] (2,-5.5) --++ (4,0) ;
\draw node [left] at (2,-5.5) {ici $z(t)<0$} ;

\pgfmathsetmacro{\ELLZERO}{2.2}
\pgfmathsetmacro{\ELLEQ}{4.1}
\pgfmathsetmacro{\AMPLITUDE}{1}
\pgfmathsetmacro{\RAYON}{0.4}
\pgfmathsetmacro{\POIDS}{1.5}
\pgfmathsetmacro{\K}{\POIDS / (\ELLEQ-\ELLZERO)}
\pgfmathsetmacro{\ELL}{\ELLEQ + \AMPLITUDE*cos(\rangle)}


% ressort seul
\begin{scope}
\ressort{\ELLZERO}{$\ell_0$}
\end{scope}

% Syst�me � l'�quilibre
\begin{scope}[xshift=3cm]%D�calage horizontal de 3 cm
\masseressort{\ELLEQ}{$\ell\e{�q}$}{$\vect{T\e{�q}}$}
\end{scope}

% Syst�me en mouvement
\begin{scope}[xshift=6cm]
\masseressort{\ELL}{$\ell(t)$}{$\vect{T}$}
\end{scope}

% L�gendes
\begin{scope}[xshift=-1cm,yshift=-5.5cm]
\draw [rounded corners=4pt,color=white,ball color=gray,smooth] (0,-2.5) circle 
(0.4) node [right=0.75cm,black] {: objet M de masse $m$} ;
\draw[ressort,decorate,rotate=0] (-0.75,-3.5)--(0.75,-3.5) node 
[right=0cm,black] {: ressort de constante de raideur $k$};
\end{scope}
\end{tikzpicture}
    }
\end{animateinline}


\end{document}
