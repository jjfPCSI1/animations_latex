%% Les valeurs utiles


\begin{animateinline}[autoplay,loop,controls]{4}
    \multiframe{36}{rangle=0+10}{

\begin{tikzpicture}[scale=1.5]
\pgfsetzvec{\pgfpointxy{-0.30619}{-0.17678}}

\pgfmathsetmacro{\THETAr}{int(35+\rangle/2)}
\pgfmathsetmacro{\PHIr}{int(40+\rangle)}

\draw (0,0) -- (1,0) ;

%\ifthenelse{\THETAr<10}{Go}{RCS}

\ifthenelse{{\THETAr > 180}}{
  \pgfmathsetmacro{\THETA}{\THETAr-180}
}{
  \pgfmathsetmacro{\THETA}{\THETAr}
}



%\ifthenelse{\PHIr > 360}{
%  \pgfmathsetmacro{\PHI}{\PHIr - 360}
%}{
%  \pgfmathsetmacro{\PHI}{\PHIr}
%}

\pgfmathsetmacro{\PHI}{60*cos(2*\rangle)}



\pgfmathsetmacro{\R}{3}
\pgfmathsetmacro{\UNIT}{0.7}
\pgfmathsetmacro{\Z}{\R*cos(\THETA)}
\pgfmathsetmacro{\X}{\R*sin(\THETA)*cos(\PHI)}
\pgfmathsetmacro{\Y}{\R*sin(\THETA)*sin(\PHI)}


\coordinate (M) at (\Y,\Z,\X) ;
\coordinate (P) at (\Y, 0,\X) ;
\coordinate (H) at ( 0,\Z, 0) ;
\coordinate (O) at ( 0, 0, 0) ;
%\coordinate (Mr) at ({(M) + \UNIT*(-sin(\THETA)*sin(\PHI),cos(\THETA),sin(\THETA)*cos(\PHI))}) ;


%\clip  (-3.1,-0.7) rectangle (3.5,3.5);

%% Les axes

\draw [->,semithick] (-4.5,0,0) -- (4.5,0,0)  node [right] {$y$} ;
%\draw [semithick] ({\R*sin(-75)},0,{\R*cos(-75)}) -- ({-\R},0,{\R}) -- ({\R},0,{\R}) 
%	-- ({\R},0,{-\R}) -- ({-\R},0,{-\R}) ;
\draw [semithick] ({(\R)*sin(-75)*sin(75)},{(\R)*cos(75)},{(\R)*cos(-75)*sin(75)})
	-- ({-\R-1},0,{\R+1}) -- ({\R+1},0,{\R+1}) 
	-- ({\R+1},0,{-\R-1}) -- ({-\R-1},0,{-\R-1}) ;
\filldraw [white] (O) circle ({\R}) ;
%\draw [semithick] ({\R*sin(-75)},0,{\R*cos(-75)}) -- ({-\R},0,{\R}) -- ({\R},0,{\R}) 
%	-- ({\R},0,{-\R}) ;
\draw [semithick] 
	   ({-\R-1},0,{\R+1}) -- ({\R+1},0,{\R+1}) 
	-- ({\R+1},0,{-\R-1});
\draw [->,semithick] (0,-4,0) -- (0,3.5,0)  node [right] {$z$} ;

% Les diff�rents m�ridiens pour voir la sph�re
\foreach \p in {-75,-60,...,90} {
  \draw [dotted] (0,\R,0)
    \foreach \t in {0,10,...,160} {
        -- ({\R*sin(\t)*sin(\p)},{\R*cos(\t)},{\R*sin(\t)*cos(\p)}) 
    } ;
}

% Et les diff�rents parall�les
\foreach \t in {0,20,...,180} {
  \draw [dotted] ({\R*sin(\t)*sin(-70)},{\R*cos(\t)},{\R*cos(-70)*sin(\t)})
   \foreach \p in {-70,-60,...,90} {
        -- ({\R*sin(\t)*sin(\p)},{\R*cos(\t)},{\R*sin(\t)*cos(\p)}) 
   } ;
}

\pgfmathsetmacro{\t}{90}
  \draw ({\R*sin(\t)*sin(-70)},{\R*cos(\t)},{\R*cos(-70)*sin(\t)})
   \foreach \p in {-70,-60,...,90} {
        -- ({\R*sin(\t)*sin(\p)},{\R*cos(\t)},{\R*sin(\t)*cos(\p)}) 
   } ;


%\draw [dashed] (0,0,{-\R-1.5}) -- (0,0,\R) ; % x
\draw [dashed] (0,0,{0}) -- (0,0,\R) ; % x
%\draw [dashed] (0,{-\R},0) -- (0,\R,0) ; % z
%\draw [dashed] ({-\R},0,0) -- (\R,0,0) ; % y
\draw [->,semithick] (0,0,\R) -- (0,0,{\R+2})  node [below] {$x$} ;
%\draw [->] (\R,0,0) -- ({\R+0.5},0,0) node [above] {$y$} ;
%\draw [->] (0,\R,0) -- (0,{\R+0.5},0) node [right] {$z$} ;
%\draw ({-\R},0,0) -- ({-0.5-\R},0,0)
%      (0,{-\R},0) -- (0,{-0.5-\R},0) ;


% Le cercle


%\filldraw [thick, fill=gray] (O) -- (H) node [pos=0.5,left] {$r\cos\theta$} 
%		-- (M) 		
%		-- (P) -- (O) 
%		;
\draw [thick] (O) -- (H) %node [pos=0.5,\ifthenelse{\PHI>0}{left}{right}] {$r\cos\theta$} 
		-- (M) 		
		-- (P) -- (O) 
		;

%\draw [semithick,dotted] (0,0,\X) -- (P) -- (\Y,0,0) ;

\coordinate (Mi) at ($ 0.7*(P) + 0.3*(O)$);
\coordinate (Mib) at ($ (Mi) + (-0.2,-0.2) $);
\coordinate (Orig) at (2.5,-2.5) ;

\draw [<-] (Mi) .. controls (0,-1.5) and (1.5,0) .. (Orig)
		node [below=-3pt] {$ r\sin\theta$} ;

\draw [<-] (-0.1,{\Z/2}) .. controls (-1.5,0) and (0,1.5) .. (-2.5,2.5)
		node [above=-3pt] {$ r\cos\theta$} ;


% Position des points principaux
%  \filldraw[black] (M) circle(1pt) node [left] {$M$} ;
  \filldraw[black] (H) circle(1pt) node [left] {$H$} ;
  \filldraw[black] (P) circle(1pt) node [below] {$P$} ;

%% Les lignes d'angles
  \draw [ultra thick,->] (O) -- (M) node [pos=0.5,below] {$\quad r$};


%% Les ellipses des construction
    
% Le m�ridien d'abord
\draw [dashed,semithick] (0,\R,0)
    \foreach \t in {0,10,...,180} {
        -- ({\R*sin(\t)*sin(\PHI)},{\R*cos(\t)},{\R*sin(\t)*cos(\PHI)}) 
    } ;

% Le parall�le ensuite
\draw [dashed,semithick] ({-\R*sin(\THETA)*sin(70)},{\R*cos(\THETA)},{\R*cos(70)*sin(\THETA)})
   \foreach \p in {-70,-60,...,90} {
        -- ({\R*sin(\THETA)*sin(\p)},{\R*cos(\THETA)},{\R*sin(\THETA)*cos(\p)}) 
   } ;
    
%  \ELLIPSEUN
%%  \ELLIPSEDEUX
%  \ELLIPSETROIS
%%  \ELLIPSEQUATRE

%% Vecteurs
%\VER
\draw [->,ultra thick,shift={(M)}] (0,0,0) -- ({\UNIT*sin(\THETA)*sin(\PHI)},{\UNIT*cos(\THETA)},{\UNIT*sin(\THETA)*cos(\PHI)}) node [right] {$\ver$};
%\VETHETA
\draw [->,ultra thick,shift={(M)}] (0,0,0) -- ({\UNIT*cos(\THETA)*sin(\PHI)},{-\UNIT*sin(\THETA)},{\UNIT*cos(\THETA)*cos(\PHI)}) node [right] {$\vetheta$};
%\VEPHI
\draw [->,ultra thick,shift={(M)}] (0,0,0) -- ({\UNIT*cos(\PHI)},{0},{-\UNIT*sin(\PHI)}) node [right] {$\vephi$};

%  \draw (0,-3) -- (0,3) ;

%% Angles



%\THETA
{
\pgfmathsetmacro{\R}{1}
% Le m�ridien d'abord
\draw [->,thick] (0,\R,0)
    \foreach \t in {0,...,10} {
        -- ({\R*sin(\t*\THETA/10.0)*sin(\PHI)},{\R*cos(\t*\THETA/10.0)},{\R*sin(\t*\THETA/10.0)*cos(\PHI)}) 
    } ;
\draw ({\R*sin(\THETA/2)*sin(\PHI)},{\R*cos(\THETA/2)},{\R*sin(\THETA/2)*cos(\PHI)})
	node [above right=-2pt] {$\theta$};

%\PHI
% Le parall�le ensuite
\draw [->,thick] ({\R*sin(90)*sin(0)},{\R*cos(90)},{\R*cos(0)*sin(90)}) 
   \foreach \t in {0,...,10} {
        -- ({\R*sin(90)*sin(\PHI*\t/10.0)},{\R*cos(90)},{\R*sin(90)*cos(\PHI*\t/10.0)}) 
   } ;
\draw ({\R*sin(90)*sin(\PHI/2.0)},{\R*cos(90)},{\R*cos(\PHI/2)*sin(90)}) node [below=-2pt] {$\ph$} ;

}

\end{tikzpicture}
}
\end{animateinline}

