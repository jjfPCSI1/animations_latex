\newcommand{\refraction}[2]{
            \pgfmathsetmacro{\angle}{#1}
            \pgfmathsetmacro{\angles}{#2}
            % L'interface
    	    \draw (-2.25,0) -- (2.25,0) ;
    	    % La normale
    	    \draw [dashed] (0,2) -- (0,-2) ;
    	    % Le rayon incident
    		\draw [thick,->,rotate={90+\angle}] (2,0)  -- (1,0) ;
    		\draw [thick,rotate={90+\angle}] (1,0) -- (0,0) ;
    		% sa continuit�
    		\draw [dotted,rotate={-90+\angle}] (2,0)  -- (0,0) ;
    		% Le rayon r�fl�chi
    		\draw [help lines,rotate={90-\angle}] (2,0) -- (1,0) ;
    		\draw [help lines,<-,rotate={90-\angle}] (1,0) -- (0,0) ;
    		% Le rayon r�fract�
    		\draw [thick,rotate={-90+\angles}] (2,0) -- (1,0) ;
    		\draw [thick,<-,rotate={-90+\angles}] (1,0) -- (0,0) ;
    		% L'angle incident
    		\draw (0,0.3) arc (90:90+\angle:0.3) ;
    		\draw [rotate={\angle/2}] (0,0.5) node {$i_1$} ;
    		% L'angle r�flact�
    		\draw (0,-0.35) arc (-90:-90+\angles:0.35) ;
    		\draw (0,-0.4) arc (-90:-90+\angles:0.4) ;
    		\draw [rotate={-90+\angles/2}] (0.63,0) node {$i_2$} ;
    		% L�gendes
    		\draw (-1.5,0) node [above] {$n_1$} ;
    		\draw (-1.5,0) node [below] {$n_2$} ;
}%
\begin{animateinline}[autoplay,loop,controls]{2}
    \multiframe{36}{iangle=60+5}{
    	\begin{tikzpicture}[declare function={
    	firstCut(\x) = (\x > 180)*(0-180) + \x ;     % Ces deux fonctions permettent de
    	secondCut(\x)= (\x > 90) *(180 - 2*\x) + \x ;% se ramener entre 0 et 90
    	sortie(\x)   = asin(sin(\x)/1.33) ;
    	}]
    		% On se ram�ne entre 0 et 90
            \pgfmathsetmacro{\ANGLE}{firstCut(\iangle)}
            \pgfmathsetmacro{\ANGLE}{secondCut(\ANGLE)}
            % Angle r�fract� en sortie
            \pgfmathsetmacro{\ANGLEs}{sortie(\ANGLE)}
            \draw (0,2.5) node [above]{$n_2>n_1$} ;
            \draw (0,2) node [above] {On se rapproche de la normale} ;
    	    \refraction{\ANGLE}{\ANGLEs}
    	    \begin{scope}[shift={(5,0)}]
            \draw (0,2.5) node [above]{$n_2<n_1$} ;
            \draw (0,2) node [above] {On s'�loigne de la normale} ;
    	    \refraction{\ANGLEs}{\ANGLE}
    	    \end{scope}
    	\end{tikzpicture}
    }
\end{animateinline}
