\begin{animateinline}[autoplay,loop,controls]{2}
    \multiframe{18}{iangle=60+10}{
    	\begin{tikzpicture}[declare function={
    	firstCut(\x) = (\x > 180)*(0-180) + \x ;     % Ces deux fonctions permettent de
    	secondCut(\x)= (\x > 90) *(180 - 2*\x) + \x ;% se ramener entre 0 et 90
    	}]
    		% On se ram�ne entre 0 et 90
            \pgfmathsetmacro{\ANGLE}{firstCut(\iangle)}
            \pgfmathsetmacro{\ANGLE}{secondCut(\ANGLE)}
            % L'interface
    	    \draw (-2.25,0) -- (2.25,0) ;
    	    \foreach \x in {-9,...,8} {
    	       \draw ({\x/4},-0.25) -- ({\x/4+0.25},0) ;
    	    }
    	    % La normale
    	    \draw [dashed] (0,2) node [above] {$N$} -- (0,-0.2) ;
    	    % Le rayon incident
    		\draw [thick,->,rotate={90+\ANGLE}] (2,0) node [above] {$I$} -- (1,0) ;
    		\draw [thick,rotate={90+\ANGLE}] (1,0) -- (0,0) node [below] {$S$};
    		% Le rayon r�fl�chi
    		\draw [thick,rotate={90-\ANGLE}] (2,0) node [above] {$R$} -- (1,0) ;
    		\draw [thick,<-,rotate={90-\ANGLE}] (1,0) -- (0,0) ;
    		% L'angle incident
    		\draw (0,0.3) arc (90:90+\ANGLE:0.3) ;
    		\draw [rotate={\ANGLE/2}] (0,0.45) node {$i$} ;
    		% L'angle r�fl�chi
    		\draw (0,0.35) arc (90:90-\ANGLE:0.35) ;
    		\draw (0,0.4) arc (90:90-\ANGLE:0.4) ;
    		\draw [rotate={-\ANGLE/2}] (0,0.63) node {$i'$} ;
    	\end{tikzpicture}
    }
\end{animateinline}
