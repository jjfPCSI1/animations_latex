\begin{animateinline}[autoplay,loop,controls]{10}
    \multiframe{36}{rangle=0+10}{
\begin{tikzpicture} [scale=0.75, decoration={coil,aspect=0.4,segment 
length=3mm,amplitude=3mm}]
%decoration : g�re l'aspect du ressort par l'instruction decorate
\tikzset{ressort/.style={thick,gray,smooth}} %d�finition d'un style de ressort

\draw [dashed] (4.5,2) --++ (0,-8);
\draw node [above] at (4.5,2) {O}; 
\draw [->] (0,2) --++ (7,0) node [above] {$x$};

\pgfmathsetmacro{\ELLZERO}{4.5}
\pgfmathsetmacro{\AMPLITUDE}{1.3}
\pgfmathsetmacro{\ELLUP}{  \ELLZERO - \AMPLITUDE*cos(\rangle)}
\pgfmathsetmacro{\ELLDOWN}{\ELLZERO + \AMPLITUDE*cos(\rangle)}
\pgfmathsetmacro{\K}{3}


%\newcommand{\ellmacro}[1]{
%\ifthenelse{\equal{#1}{\ELLZERO}}{\ell_0}{\ell(t)}
%}

% #1 -> la longueur du ressort
% #2 -> \ell_0 ou \ell(t)
\newcommand{\systemeMasseRessort}[3]{

%bloc qui tient le ressort 
\draw[thick,gray] (0,0) -- (0.3,0); 
\fill [pattern=north east lines] (0,0.5) rectangle (-0.3,-0.5); 
%pattern d�finit un style de remplissage de la forme
\draw[thick](0,0.5)--(0,-0.5); %fin ressort
% Le ressort proprement dit
\pgfmathsetmacro{\taille}{1.2*#1}
\draw[ressort,decorate,decoration={coil,aspect=0.62,segment length = \taille mm,amplitude=3mm}] 
(0.3,0)--(#1,0) ;
% La l�gende au-dessus
\draw[line width=0.5pt,<->](0,+0.8) -- (#1,+0.8) node [midway,above] {$#2$} ;
% La masse
\draw [rounded corners=4pt,color=white,ball color=gray,smooth] (#1,-0.5) rectangle ({#1+1},0.5);
\draw[thick](2,-0.5)--(7.3,-0.5); %fin du plancher
\fill [pattern=north east lines] (2,-0.5) rectangle (7.3,-0.8); 
\draw [very thick,-latex] ({#1+0.5},0)--++(0,-1.5) node [right] {$\vect{P}$};
\draw [very thick,-latex] ({#1+0.6},-0.5)--++(0,1.5) node [right] {$\vect{N}$};
% Si on est � \ell_0, pas de fl�che, sinon on la marque
\ifthenelse{\equal{#1}{\ELLZERO}}{
\draw (#1,0) node [above] {$#3$} ;
}{
\draw [very thick,-latex] (#1,0)--++({-\K*(#1-\ELLZERO)},0) node [midway,above] {$\vect{T}$};
}
}


\begin{scope}[yshift=0cm]
\systemeMasseRessort{\ELLUP}{\ell(t)}{\vect{T}}
\end{scope}

\begin{scope}[yshift=-3cm]
\systemeMasseRessort{\ELLZERO}{\ell_0}{}
\end{scope}

\begin{scope}[yshift=-6cm]
\systemeMasseRessort{\ELLDOWN}{\ell(t)}{\vect{T}}
\end{scope}
\end{tikzpicture}
}
\end{animateinline}
