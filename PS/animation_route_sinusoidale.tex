\newcommand{\ALTITUDE}[2]{\amplitude*cos(360*#1/2.5 - #2)}
\newcommand{\VOITURE}[3]{
% #1 Abscisse roue 1
% #2 amplification
% #3 d�phasage
\pgfmathsetmacro{\roue}{0.5}
\pgfmathsetmacro{\decalage}{2.5}
\pgfmathsetmacro{\XROUEDEUX}{#1+\decalage}
\pgfmathsetmacro{\XV}{#1-1.25}
\pgfmathsetmacro{\V}{#2}
\pgfmathsetmacro{\VZERO}{3}
\pgfmathsetmacro{\Q}{10}
\pgfmathsetmacro{\ranglephi}{\rangle+#3}
\pgfmathsetmacro{\YV}{\roue + #2*\ALTITUDE{#1}{\ranglephi} }

    \shade[top color=red, bottom color=white, shading angle={135}]
        [draw=black,fill=red!20,rounded corners=1.2ex,very thick] (\XV,\YV) -- ++(0,1) -- ++(1,0.3) --  ++(3,0) -- ++(1,0) -- ++(0,-1.3) -- (\XV,\YV) -- cycle;
    \draw[very thick, rounded corners=0.5ex,fill=black!20!blue!20!white,thick]  (\XV+1,\YV+1.3) -- ++(1,0.7) -- ++(1.6,0) -- ++(0.6,-0.7) -- (\XV+1,{\YV+1.3});
    \draw[thick]  (\XV+3.2,{\YV+1.3}) -- (\XV+3.2,{\YV+2});
    \ROUE{\roue}{#1}{{\roue+\ALTITUDE{#1}{\rangle}}} 
    \ROUE{\roue}{\XROUEDEUX}{{\roue+\ALTITUDE{\XROUEDEUX}{\rangle}}} 
}
\newcommand{\ROUE}[3]{
% #1 -> rayon externe
% #2 -> abscisse centre
% #3 -> altitude centre
\draw [draw=black,fill=gray!50,thick] (#2,#3) circle (#1) ;
\draw [draw=black,fill=gray!80,semithick] (#2,#3) circle ({0.8*#1}) ;
}
\newcommand{\graphiqueVitesse}[3]{
% #1 -> vitesse (1, 3 ou 9), r�sonance � 3
% #2 -> titre
% #3 -> rangle
\begin{tikzpicture}
\pgfmathsetmacro{\amplitude}{0.2}
\draw [smooth,samples=100,domain=0:5.5] plot(\x,\ALTITUDE{\x}{#3}) ;%{\amplitude*cos(360*\x/2.5 - #3)}) ;
\draw (2.75,3.5) node {#2};

\VOITURE{1.5}{\AMPLIFICATION}{\DEPHASAGE}
\end{tikzpicture}
}
%
\begin{animateinline}[autoplay,loop,controls]{8}
    \multiframe{36}{rangle=00+10}{
\pgfmathsetmacro{\AMPLIFICATION}{1}
\pgfmathsetmacro{\DEPHASAGE}{0}
\graphiqueVitesse{1}{Basses vitesses}{\rangle}
}
\end{animateinline}
\begin{animateinline}[autoplay,loop,controls]{16}
    \multiframe{36}{rangle=00+10}{
\pgfmathsetmacro{\AMPLIFICATION}{2}
\pgfmathsetmacro{\DEPHASAGE}{270}
\graphiqueVitesse{3}{R�sonance}{\rangle}
}
\end{animateinline}
\begin{animateinline}[autoplay,loop,controls]{36}
    \multiframe{36}{rangle=00+10}{
\pgfmathsetmacro{\AMPLIFICATION}{0.1}
\pgfmathsetmacro{\DEPHASAGE}{190}
\graphiqueVitesse{9}{Hautes vitesses}{\rangle}
}
\end{animateinline}
